% !TEX TS-program = pdflatex
% !TEX encoding = UTF-8 Unicode

% This is a simple template for a LaTeX document using the "article" class.
% See "book", "report", "letter" for other types of document.

\documentclass[11pt]{article} % use larger type; default would be 10pt

\usepackage[utf8]{inputenc} % set input encoding (not needed with XeLaTeX)

%%% Examples of Article customizations
% These packages are optional, depending whether you want the features they provide.
% See the LaTeX Companion or other references for full information.

%%% PAGE DIMENSIONS
\usepackage{geometry} % to change the page dimensions
\geometry{a4paper} % or letterpaper (US) or a5paper or....
% \geometry{margin=2in} % for example, change the margins to 2 inches all round
% \geometry{landscape} % set up the page for landscape
%   read geometry.pdf for detailed page layout information

\usepackage{graphicx} % support the \includegraphics command and options

% \usepackage[parfill]{parskip} % Activate to begin paragraphs with an empty line rather than an indent

%%% PACKAGES
\usepackage{booktabs} % for much better looking tables
\usepackage{array} % for better arrays (eg matrices) in maths
\usepackage{paralist} % very flexible & customisable lists (eg. enumerate/itemize, etc.)
\usepackage{verbatim} % adds environment for commenting out blocks of text & for better verbatim
\usepackage{subfig} % make it possible to include more than one captioned figure/table in a single float
% These packages are all incorporated in the memoir class to one degree or another...

%%% HEADERS & FOOTERS
\usepackage{fancyhdr} % This should be set AFTER setting up the page geometry
\pagestyle{fancy} % options: empty , plain , fancy
\renewcommand{\headrulewidth}{0pt} % customise the layout...
\lhead{}\chead{}\rhead{}
\lfoot{}\cfoot{\thepage}\rfoot{}

%%% SECTION TITLE APPEARANCE
\usepackage{sectsty}
\allsectionsfont{\sffamily\mdseries\upshape} % (See the fntguide.pdf for font help)
% (This matches ConTeXt defaults)

%%% ToC (table of contents) APPEARANCE
\usepackage[nottoc,notlof,notlot]{tocbibind} % Put the bibliography in the ToC
\usepackage[titles,subfigure]{tocloft} % Alter the style of the Table of Contents
\renewcommand{\cftsecfont}{\rmfamily\mdseries\upshape}
\renewcommand{\cftsecpagefont}{\rmfamily\mdseries\upshape} % No bold!

%%% END Article customizations
\usepackage{hyperref}
\usepackage{nameref}
\usepackage{footnote}
\usepackage{tipa}
\usepackage{amsthm}


\newcommand{\linkto}[1]{\textbf{\nameref{#1}}}
\makesavenoteenv{tabular}

%%% The "real" document content comes below...

\title{Turnabout Awakening}
\author{Patrick Martin}
%\date{} % Activate to display a given date or no date (if empty),
         % otherwise the current date is printed 

\begin{document}
\maketitle
\begin{center}
\includegraphics[scale=0.7]{archanea_colored.png}
\end{center}
\newpage
\tableofcontents
\newpage

\section{Introduction}

Twenty-two years ago, in the year 63 Mostyn, the Exalt of \linkto{nations:ylisse} declared a crusade against \linkto{nations:plegia}, vowing to rid the continent of the Grimleal (followers of the dragon \linkto{religion:grima}), whom Ylissean religion deems to be evil agents of discord and destruction. However, Ylisse failed to progress very far into Plegian territory, and the war stood at a stalemate for twelve long years. It was at that point, ten years ago in 75 Mostyn, that the Exalt suddenly died and his daughter, \linkto{people:emmeryn}, took the throne. A compassionate ruler, Emmeryn quickly negotiated a white peace with Plegia's \linkto{people:gangrel} and focused her attention to healing the realm from the scars of war. Plegians remain skeptical of the new Exalt's compassion, and harbor deep hatred towards all Ylisseans in the wake of that war.

The \linkto{nations:ferox} to the north is a collection of kingdoms, divided in loyalty to the Eastern and Western Khanates, which are both sovereign under the Feroxi throne. In Ferox, strength is the most valued quality of all, and politics revolves around posturing oneself as stronger than the alternatives. Because of this, little attention is paid to the goings-on outside Feroxi borders, and thus Ferox is able to maintain warm relations with both Ylisse and Plegia. 

The starting date is 10 Emmeryn, Ket 21 (Pentek).

\subsection{Losokyo}

Losokyo, like any large city, has been plagued with thieves and pickpockets for time immemorial. However, soon after the Ylissean Crusade began to truly wear on the Halidom, about two years into the war, criminal elements began taking over large sections of the city. In response, the city took drastic measures to combat the crime problem and suspended the principle of ``innocent until proven guilty''. The local priestesses of \linkto{religion:themis}, the dragon of Justice, fearing their god's retribution, rapidly trained prosecutors, defense attorneys, and judges in an attempt to ensure justice is carried out even under these dire circumstances.
% nd, due to the stress this caused the legal system, enacted a maximum time limit on court cases and
\begin{center}
``The end is only justified through proper means'' \\
- Inscription on the Central Courthouse
\end{center}

The priestesses maintained this practice for a couple years, until 68 Mostyn, when they left Losokyo to establish the Themis Legal Academy in a more central location of the Halidom, in order to train legal experts for use across the realm. However, with their departure, the people of Losokyo have gradually lost faith in this new legal system, particularly becoming wary of corrupt prosecutors fabricating evidence and police incompetence. Among several events, two in particular stand out as embarrassing failures of justice.

\begin{enumerate}
\item \linkto{events:dl6}\\
During this period of prosecutorial corruption, public faith in the law rested entirely on the defense attorneys. In 73 Mostyn, one of the most famous defense attorneys, \linkto{people:gregory}, took on master prosecutor \linkto{people:mvonkarma} in a high profile case. During a recess in the trial, Gregory was in a lift with his son, \linkto{people:miles}, and a court bailiff, \linkto{people:yogi}, when an earthquake struck, incapacitating the lift for a period of time. When the lift was repaired and opened, Gregory was found dead, shot by a gun.

The police were unable to make any progress on this case, and so secretly consulted \linkto{people:mistyfey}, a spirit medium and head of \linkto{places:kurain}. Misty channeled Gregory's spirit, who fingered Yogi as the culprit. The fact that the police consulted a spirit medium was leaked to the press, causing a scandal. Further, Yogi's attorney, \linkto{people:hammond}, succeeded on acquitting his client through pleading temporary insanity. 

\item \linkto{events:kidnapdh}\\
Not too long after the DL-6 incident, \linkto{people:fawles} kidnapped \linkto{people:dahlia}, the daughter of a wealthy jeweler, and demanded a \$250,000 GP diamond for her return, to be delivered to \linkto{places:duskybridge}. \linkto{people:valerie}, Dahlia's sister and a police officer, was chosen to conduct the random payment, however Fawles pushed Dahlia off the side of the bridge midway through. Dahlia fell into the dangerous current to almost certain death, and the gem was lost as well. For this, Fawles was sentenced to death, however as of Emmeryn 3, his sentence has not been carried out.
\end{enumerate}

Under the peace brought to the halidom by Exalt \linkto{people:emmeryn}, Losokyo has seen only a slight return to order. In recent months, police have been flummoxed by a series of murders that they have been unable to solve. Outside the city, bandits terrorize merchants, who are nearly defenseless after the Exalt's downsizing of the Ylissean army. Losokyan militas have scoured the countrysides in search of bandit camps, but have made little progress. Unbeknownst to the civillians, many of these bandits are actually Plegian raiding parties, directed by \linkto{people:gangrel} in attempts to re-ignite war between Ylisse and Plegia.

\subsection{Ylisse}
Recognizing the need for her halidom to heal after the trials of the Crusade, Exalt \linkto{people:emmeryn} has turned her attention to internal affairs. She all but disbanded the Ylissean army, allowing those soldiers to return to their villages, which had suffered from a lack of manpower to tend to farms and workshops. Instead, Ylisse's internal affairs are protected by the Shepherds, a royal guard led by Prince \linkto{people:chrom} that travels across the realm to protect against bandits and other threats. While effective when present, the Shepherds are unable to protect the entire halidom at once, and bandit attacks have only grown in frequency. This, however, is also largely due to Plegian raiding parties that disguise themselves as bandits to terrorize Ylissean towns. The Shepherds are aware of the true identity of these ``bandits'', however Emmeryn is adamant in avoiding another war.


\subsection{The World}
\subsubsection{Time}
Years in Ylisse are marked as years since the current Exalt's coronation, which always occurs on Ich 1 (although they have \textit{de facto} power as soon as their predecessor steps down), and starts counting at 1. For this reason, Ich 1 is also usually referred to using the current Exalt's name when not ambiguous; when referring to the current Exalt, it is typical to use their title. The starting year is thus Emmeryn 3, as Exalt Emmeryn took the throne three years ago. If the speaker wishes to emphasize historical continuity, they may use Naga years, counting the years since the first Awakening ritual. Emmeryn 3 corresponds to Naga 1012. Naga years are also used across cultures, although they are referred to as ``Grima years'' in Plegia, and in Cohdopia one would say ``1012 Naga''. Each year is broken into eleven months:

\begin{center}
\begin{tabular}{l l l}
1. Ich & 2. Ket & 3. Ser \\
4. Yaun\footnote{[\textipa{jon}]} & 5. Chine\footnote{[\textipa{tSAIn}]} & 6. Haut\footnote{[\textipa{hot}]} \\
7. Eb & 8. Neige\footnote{[\textipa{neZ}]} & 9. Kyu \\
10. Tiz & 11. Zaun\footnote{[\textipa{zon}]} \end{tabular}
\end{center}

Each month consists of four weeks, each thereof consisting of eight days, which are

\begin{center}
\begin{tabular}{l l l l}
1. Neiged\footnote{[\textipa{"ne.dZEd}]} & 2. Kedd & 3. Serda & 4. Chutork \\
5. Pentek & 6. Sombat & 7. Vasarnap & 8. Honapon \end{tabular}\end{center}

Days are divided into 24 hours, typically referred to in 12-hour format and disambiguated with AM/PM. Hours consist of 60 minutes, etc. 

\section{The First Turnabout}
The party enters Losokyo expecting to meet \linkto{people:miafey}, however they are met by \linkto{tbd}, who welcomes them and informs the party that Mia is busy and so they were asked to show them around. TBD guides them to the center of the city, pointing out the central Court House and the clock tower at the top. 
\begin{center}
``It truly is a wonderful chime. I've lived here for nigh on a decade and still enjoy its song every hour.''
- TBD \end{center}

After answering any questions the party may have about the city, TBD then directs them towards the Gatewater Hotel, where rooms have already been reserved for the party. Shortly after entering, however, TBD leaves the party to exit the hotel in order to hear the 2PM chime of the clock tower, inviting the party out to hear it.

\begin{center}
``This hotel is special; you can't hear the clock tower from inside. I figure they don't want it disturbing the guests if they're trying to nap midday or something. This hotel serves plenty of foreign tourists, who may suffer from tele-lag.''
\end{center}

Upon re-entering the hotel, \linkto{people:larry} walks by, possibly unnoticed. If noticed, he is clearly annoyed and unwilling to talk to anyone. TBD then leads the party to their room, a suite with N bedrooms on the side of a large meeting room on the fourth floor. There is a balcony, which looks over Fey \& Co. Law Offices. When the lighting is right (say, at night), you can see right through the window into the main office. TBD answers any questions the party has about the accomodations and the city, and further informs them that Mia needed to defend a client in court on a very short notice, and so won't be available until later tonight.

About two hours later, if the party is in a public place, Detective \linkto{people:gumshoe} and the Losokyan police will arrive and arrest one of the party members for murder, and take that member to the Detention Center, and the trial will be tomorrow. Alternatively, if Larry is a close friend of any of the party members, he will be arrested instead. If the party figures out and attempts to investigate the crime scene, they will mostly be stonewalled, as they are not lawyers and thus aren't allowed to investigate. However they might learn the following information from talking to Detective Gumshoe:
\begin{itemize}
\item The victim's name was Cindy Stone
\item Cindy was murdered via blunt force trauma
\item Cindy had just returned from Babahl
\end{itemize}

If the party goes to the detention center and the suspect is Larry, he will be confused. Earlier that day he went to Gatewater Hotel to visit his girlfriend, but she wasn't home, so he left and wandered around for the rest of the day. If pressed, he will say his girlfriend's name is Cindy Stone. If he finds out that Cindy Stone was murdered, he goes insane, blabbering about how his life is over, nothing matters anymore, he might as well get a guilty verdict, etc. 

\begin{center}
``Oh, it's all over... I... I'm finished. Finished! I can't live in a world without her! I can't! Who... who took her away from me? Who did this!? Aww, ya gotta tell me! Who took my baby away!?''
\end{center}

At some point, the party will be asked who will be defending the suspect. The only acceptable answer is for the party themselves (or a subset thereof) to do the defense, as they will not be able to meet with Mia before the deadline, and no lawyers they may find will be willing to take on a client with such short notice.

That night, Mia will visit the party and tell them she heard about what happened (if the suspect is not Larry), and asks if they managed to find a lawyer. Since the party has already registered themselves as the defense, Mia won't be able to be the attorney, however she will offer to stand at the bench as an assistant. 

\subsection{Trial}
The trial begins at 10 AM, although the group and the defendant meet in Defendant Lobby No. 2 to prepare for the case. If Larry was not the culprit, the bailiff will approach the party and inform them that due to new, conclusive evidence, the suspect in this case is now Larry Butz. Larry pleads with the party to save him, and Mia encourages the party to continue the defense, as he doesn't seem guilty. Once they have agreed, however, Larry will become incoherent, and the group continues discussing the case until the trial begins. Either Mia or the Bailiff will bring the party Cindy's autopsy report. In the Courtroom, the \linkto{people:judge} will note that the defense team is new, and give a short test to ascertain their readiness:
\begin{itemize}
\item Please state the name of the defendant in this case. (If multiple choice: Party Member, Larry Butz, Mia Fey)
\item This is a murder trial; tell me, what wsa the victim's name? (If multiple choice: Mia Fey, Cinder Block, Cindy Stone)
\item What was the cause of death? She died because she was... (If multiple choice: Poisoned, hit with a blunt object, strangled)
\end{itemize}

The Judge then invites the prosecutor, \linkto{people:wpayne}, to explain what the murder weapon was: \linkto{items:thinker}, found lying on the floor, next to the victim. The prosecutor then calls Larry to the stand and asks some questions.

\subsubsection{Witness Testimony: Larry Butz}
\begin{center}
\begin{tabular}{p{2.5in} p{2.5in}}
Mr. Payne & Larry \\\hline
Ahem, Mr. Butz. Is it not true that the victim had recently dumped you? & Hey, watch it buddy! We were great together! We were Romeo and Juliet, Cleopatra and Mark Anthony! I wasn't dumped! She just wasn't responding to any of my letters! Or seeing me.. Ever. WHAT'S IT TO YOU, ANYWAY!? (continue from transcript, until ``Duuude!'')\\
We can clearly see what kind of woman this Ms. Stone was. Tell me, Mr. Butz, what do you think of her now? \\
\end{tabular}
\end{center}

At this point, Mia turns to the party and tells them that it would probably not be good to let Larry answer that question, although whether the party lets him or not, he will.
blahblah


\begin{center}
\begin{tabular}{p{2.5in} p{2.5in}}
Mr. Payne & Larry \\\hline
I believe the accused's motive is clear to everyone. Next queston! You went to the victim's apartment on the day of the murder, did you not? & Heh? Heh heh. Well, maybe I did, and maybe I didn't!
\end{tabular}
\end{center}

The Judge turns to the party, asking them to advise their client to answer the question. Regardless of whether they do, Mr. Payne interjects that he has a witness proving Larry did go to the victim's apartment that day. 

\begin{center}
``On the day of the murder, my witness was selling newspapers at the victim's building. Please bring Mr. Frank Sahwit to the stand!''
\end{center}

\subsubsection{Witness Testimony: Frank Sahwit 1}
\begin{center}
\begin{tabular}{p{4in}}
I was going door-to-door, selling subscriptions when I saw a man fleeing an apartment.\\
I thought he must be in a hurry because he left the door half-open behind him. \\
Thinking it strange, I looked inside the apartment. Then I saw her lying there... A woman... not moving... dead! \\
I quailed in fright and found myself unable to go inside. \\
I left to notify the police immediately! \\
I went to a nearby park and found a patrolling officer. \\
I remember the time exactly: It was 11:00 AM. \\
The man who ran was, without a doubt, the defendant sitting right over there. \\
\end{tabular}
\end{center}

Once the statement is over, the Judge asks Mr. Payne why the witness had to go all the way to a park to find a police officer, to which Mr. Payne informs the Judge that the same enchantments used to mute the clock tower occasionally disrupt police communications, and so they prefer not to patrol near the hotel. Satisfied, the Judge allows the party to begin their cross-examination. Mia, excited, tells the party that this is their chance to expose the lies in the witness's statement, and gives an introduction to the courtroom mechanics.

\begin{center}
``First, find contradictions between the evidence and the witness's testimony. Then, once you've found the contradicting evidence... present it and rub it in the witness's face!''
\end{center}

The contradiction is between the reported time (11:00 AM) and the time of death in the autopsy report (around 2-3 PM).

\subsection{Evidence}
\subsubsection{Cindy's Autopsy Report}
Time of death: 2-3 PM, Ket 21. Cause of death: loss of blood due to blunt force trauma. 


\subsubsection{Thinker Statue}
A statue in the shape of ``The Thinker''. It's rather heavy.

\subsubsection{Passport}
Cindy Stone apparently returned from Babahl on Ket 20, the day before the murder.
\section{Bandits of Eagle Mountain}

Over dinner with Mia, the party learns that they have been assembled to try and restore faith in law and order in Losokyo. Mia gives a brief history of the rise and fall of the Losokyan legal system. If pressed, she can also describe \linkto{events:dl6} and \linkto{events:kidnapdh}.
Talking with people around town about the bandit problem, the party learns that the city has almost been under seige for the last year because of bandits. Trade caravans need to have substantial escorts, as the bandits always seem to know 

\section{Religion}
\subsection{Dragon gods}
\subsubsection{Anankos}
\label{religion:anankos}

\subsubsection{Eitri}
\label{religion:eitri}
Ancient divine dragon of artisans. Has a sect in Losokyo that practices black magic.

\subsubsection{Grima}
\label{religion:grima}

\subsubsection{Naga}
\label{religion:naga}

\subsubsection{Themis}
\label{religion:themis}


\section{People}
\subsection{Plegia}
\subsubsection{King Gangrel}
\label{people:gangrel}

\subsubsection{Validar}
\label{people:validar}
A human illusionist and head of the Grimleal. Years of religious loyalty to Grima and practice of dark magic has twisted his form, and so he now permanently dons an illusion to look like a tiefling to mask his horrid figure. The source of this illusion is his holy symbol, a \linkto{items:grimanecklace}, which he always wears under his cloak.


\subsubsection{TSLIL}
\label{people:tslil}
A fellblood bearing the Mark of Grima on the back of their right hand. Son of \linkto{people:validar} and \linkto{people:mrstslil}, he bears both the proper Heart and Blood of Grima to be the Fell Dragon's vessel and aid in its ressurection. Unwilling to sacrifice her son for this purpose, MOTHER took TSLIL and fled to Ylisse, and paid for an illusion to hide his Mark. 

\subsection{Ylisse}
\subsubsection{Chrom}
\label{people:chrom}


\subsubsection{Dahlia Hawthorne}
\label{people:dahlia}

\subsubsection{Emmeryn}
\label{people:emmeryn}


\subsubsection{Gregory Edgeworth}
\label{people:gregory}

\subsubsection{Manfred von Karma}
\label{people:mvonkarma}

\subsubsection{Maya Fey}
\label{people:mayafey}

\subsubsection{Miles Edgeworth}
\label{people:miles}

\subsubsection{Mia Fey}
\label{people:miafey}

\subsubsection{Misty Fey}
\label{people:mistyfey}

\subsubsection{Mr. Briney}
\label{people:briney}

\subsubsection{Robert Hammond}
\label{people:hammond}

\subsubsection{Terry Fawles}
\label{people:fawles}

\subsubsection{Thalassa Gramarye}
\label{people:thalassa}

\subsubsection{Valerie Hawthorne}
\label{people:valerie}

\subsubsection{Yanni Yogi}
\label{people:yogi}

\section{Places}

\subsection{Borginia}
\label{nations:borginia}

\subsection{Cohdopia}
\label{nations:cohdopia}
Split into two nations, Allebahst and Babahl, in Naga 1002, 

\subsubsection{Allebahst}
\label{nations:allebahst}
\subsubsection{Cohdopian Normal School}
\label{places:cns}
Unlike the \linkto{places:arcaneinstitute}, the Cohdopian Normal School produces items in which magic is only a supporting element, if it exists at all. While neither school really produces weapons, an alumnus of CNS, Cole Grande, was the first to construct a gun some 75 years ago, which has been refined ever since to the variety of guns we have today. Further, while the Institute's magic is primarily of the \textit{evocation} and \textit{illusion} schools, the CNS largely specializes in \textit{transmutation} and \textit{abjuration}.
\subsubsection{Babahl}
\label{nations:babahl}
National symbol is a butterfly, called a ``H\=ato''.

\subsection{Plegia}
\label{nations:plegia}

\subsubsection{Grimstake}
\label{places:grimstake}

\subsection{Regna Ferox}
\label{nations:ferox}

\subsubsection{Ferox}
\label{places:ferox}

\subsection{Ylisse}
\label{nations:ylisse}

\subsubsection{Arcane Institute}
\label{places:arcaneinstitute}
The gem of Ivy University and central to the Losokyan way of life, the Arcane Institute develops items that allow everyone to benefit from the conveniences of magic. Some of the Institute's most famous inventions include
\begin{itemize}
\item Sender blocks and the Red network, which allow for instantaneous communication for any two people within the network\\
The price of a sender block depends on its charge capacity and recharge rate, typically
\[ price = 3 * capacity + 2*recharge \]
The most common models are the (1,1d1) for 5gp, (5,1d4) for 20gp, (10,1d6) for 35gp, and (20,3d6) for 80 gp. 
\item Lights\\
These come in various sizes, intensities, and colors, although their price increases the further from ``baseball-sized 20' range white''. These also require \textit{serifs} from a magic font to run. Many buildings run a connection to the main font in the Institute, but portable fonts and font generators are also available.
\item Portable fonts, which are organized into how much \textit{serifs} they contain. In increasing volume, the primary sizes are ``Q'', ``V'', ``N'', and ``F''. 
\item Cameras, onamonapoetically called ``Shashin''; these print their pictures after a short delay (think polaroid)
\item \st{Action} Video cameras, however they can typically only record in greyscale and require \textit{serifs} to run.
\end{itemize}

\subsubsection{Dusky Bridge}
\label{places:duskybridge}


\subsubsection{Eagle Mountains}
\label{places:eaglemountains}
A mountain range opposite Losokyo on Yam Lake. Home to Hazakura Temple and Dusky Bridge, however it has been mostly abandoned since the ``murder'' of Dahlia Hawthorne, with one notable exception being the murder of Valerie Hawthorne. Because of this, while the road to Dusky Bridge is still followable, the area away from the temple is still fairly wild, with a base 20\% chance of encountering hostile wildlife during a given period of time. Beyond the temple, the mountains are much more untamed, with a base 40\% chance of encounter.  In such an instance, roll a d20 to determine what appears:\\
\begin{center}
\begin{tabular}{c c c}
Roll & Outskirts & Heart \\\hline
1 & Swarm of Poisonous Snakes  & Goblin party \\
2-5 & Dire wolves & Black Bears \\
6-10 & Black Bears & Giant Insects \\
11-15 & Giant insects & Swarm of Ravens \\
16-20 & Swarm of Bats  & Swarm of Insects \end{tabular}\end{center}

\subsubsection{Kurain Village}
\label{places:kurain}

\subsubsection{Losokyo}
\label{places:losokyo}
A large city on Gourd Lake that serves as a trade connection between the interior of Ylisse and the ocean and capital, \linkto{places:ylisstol}. 

\subsubsection{Monastery of Highsleaux}
\label{places:highsleaux}
Established by the \textit{conjunto} Wayreen shortly after the fall of \linkto{religion:grima}, the Monastery safeguards the knowledge of the Awakening rite and tracks the locations of the sacred Gemstones. Some thirty years ago, a monk tracking the location of Sable encountered Lide and her son, \linkto{people:tslil}. Recognizing the danger posed by the child, the monk worked with Lide to save her child from his fate, staging an Orc attack to buy her time to escape to Highsleaux. The monk, however, perished at Validar's hand during the fight.

\subsubsection{Ylisstol}
\label{places:ylisstol}

\subsection{Zheng Fa}
\label{nations:zhengfa}

\section{Events}
\subsection{The DL-6 Incident}
\label{events:dl6}


\subsection{Kidnapping of Dahlia Hawthorne}
\label{events:kidnapdh}
\section{Items}
\subsection{Babahlese Ink}
\label{items:ink}
A vial of the special ink from Babahl. Burns with distinctive green flames. Provides advantage on skill checks when used to forge documents or conterfeit currency, and reduces the cost of copying spells by 10gp per level. This vial contains enough ink for 10 pages of documents, 10 levels of spells, or 50 bills.

\subsection{Boots of Mysterious Stepping}
\label{items:boots}
A \textit{cursed} item that allows the wearer to cast \textit{Misty Step} once per short rest. When cast within 30 feet of a hostile creature, the boots instead teleport the wearer to immediately in front of the most dangerous creature in range.

\subsection{Emblem of Mini/Buckle of Wumbo}
\label{items:wumbo}
A magical item in the shape of a letter ``M'', made of basalt. Contains 1 charge each of \textit{reduce} and \textit{enlarge}, which are cast by holding the item so that it looks like an ``M'' for \textit{reduce}, and upside down (``W'') for \textit{enlarge}. Casting \textit{identify} on this artifact reveals only half of its properties, corresponding to the orientation of the object. Each charge has a 25\% chance of recharging each dawn.

\subsection{Falchion}
\label{items:falchion}
A greatsword that together with the \linkto{items:fireemblem} are the Ylissean royal treasures. Forged from one of  \linkto{religion:naga}'s fangs, the sword can be empowered through the \linkto{spells:awakening} ritual.

\subsection{Figurine of Wonderous Power (seagull)}
\label{items:peeko}
A statuette of a seagull, carved out of driftwood. Acts as a normal \textit{Figurine of Wonderous Power}, and when used allows the user to cast \textit{Animal Messenger} on it. The creature contained in the figurine is a \textit{fey} seagull, previously the familiar of a now-dead Sorceress that was bound to the figurine when it was passed to the widower (\linkto{people:mrbriney}). As such, the seagull, Peeko, is primarily interested in returning to Mr. Briney, however she will still follow the commands of the owner of the figurine.\\
\\
When used by Mr. Briney, the seagull may exist in physical form indefinitely.

\subsection{Honeycomb necklace}
\label{items:honeycomb}
A \textit{cursed} necklace with a small piece of honeycomb attached. When worn, gives wearer disadvantage on all checks involving items containing the letter ``B'', and if their armor contains the letter ``B'', attacks against them have advantage. 


\subsection{Lide's Earring}
\label{items:lideearring}
A \textit{magic} earring given to \linkto{people:tslil} that, through \textit{Disguise Self}, conceals his Mark of Grima on his right hand.

\subsection{Necklace of Grima}
\label{items:grimanecklace}
A \textit{magic} necklace with a pendant of the Mark of Grima. When worn, the bearer takes the form of a \textit{Tiefling} (and looks exactly like \linkto{people:validar}'s Tiefling form), as if under the effects of \textit{Alter Self}.

\subsection{Pizza Dog}
\label{items:pizzadog}
A \textit{magic} trinket with an engraving of a dog, about 4 inches long, with circular marks on it (pepperoni) and the engraving surrounded by a rounded border (crust). When heated to 425 degrees Fahrenheit (for example, through \textit{Heat Metal}, allows the casting of \textit{Silence}. The item has 3 charges, using it in this way expends one charge, and it regains 1d4 charges every morning.

\subsection{Shield of Seals}
\label{items:fireemblem}
A shield that together with the \linkto{items:falchion} are the Ylissean royal treasures. The shield has five insets for its Gemstones: Argent, Gules, Azure, Vert, and Sable; however, only Argent is currently present. Allows one with Exalted blood to perform the \linkto{spells:awakening} ritual, although the full complement of gemstones are required for maximum effect.

\subsection{Spade of Detection}
\label{items:detectspade}
A \textit{magic} broach made of purple jade and in the form of a Spade. When within five feet of a source of magic (other than itself or other \textit{detection} items), it lets off a faint glow.

\subsection{Thalassa's Bracelet}
\label{items:bracelet}
A magical bracelet made of a unique metal alloy that fits its bearer's wrist perfectly, and thus feels tight if the bearer is tense. Grants $+2$ to Insight and Perception checks on targets the bearer can see.\\
\\
If the bearer is a descendent of \linkto{people:thalassa}, this item alerts the bearer when someone in the field of view is actively lying or hiding something from you, and further grants advantage to Insight checks in this situation.

\subsection{Thinker Statue}
\label{items:thinker}
A stone statue of ``The Thinker''. It is secretly also a clock, which announces the time in \linkto{people:larry}'s voice as ``I think it's \{TIME\}'' when its head is turned. \\
\\
The statue works because it is enchanted with \textit{Magic Mouth}, which when activated reads the time from a clock inside the base of the statue.

\subsection{Magatama}
\label{items:magatama}
A jade, comma-shaped bead that can store spiritual power. Casting \linkto{spells:empower} fills the Magatama with charges, after which it cannot be filled again for 1d3 weeks. While empowered, the Magatama glows with a faint green color, and if touching the skin of the person it is attuned to, will vibrate and expend a charge when someone within 5 feet is actively lying or hiding something from you.
\section{Spells}
\subsection{Empower}
\label{spells:empower}
\begin{tabular}{c|c|c|c}
Level & Casting Time & Range/Area & Components \\\hline
1 & 1 Action & Touch & S \\
\\
Duration & School & Attack/Save & Damage/Effect \\\hline
Instantaneous & Divination & None & Buff \end{tabular}\\
Fills a \linkto{items:magatama} with 2d6 charges of spiritual energy, allowing it to function.\\
\\
When cast at a higher level, the number of charges restored increases by 1d6 for each slot level above 1st. 

\subsection{Channel Dead}
\label{spells:channel}
\begin{tabular}{c|c|c|c}
Level & Casting Time & Range/Area & Components \\\hline
2 & 5 Minutes & Touch & S, M* \\
\\
Duration & School & Attack/Save & Damage/Effect \\\hline
(C) 1 Hour & Divination & None & Shapechanging \end{tabular}\\

You kneel and move your hands in an arcane pattern, extending your consciousness into the abyss. Using the name and face of the deceased, you find their spirit and, if the spirit is willing, the spirit returns to the world of the living, using your body as a vessel. Your body and clothes adjust to match the form of the deceased before their death, although some characteristics, such as hair color, do not change. Your consciousness is suppressed during this time, as you have no connection to the physical world, except through your concentration of this spell. As such, your only possible action is to end the spell.

Channeling a spirit of level higher than the casting level of this spell requires an ability saving throw using your spellcasting ability. The DC is 10 + the spirit's level. On a failure, you are unable to end the spell early, and are unable to use this spell again for 1d4 days.

If the targeted spirit is being channeled at the time of casting, the two casters contest control by rolling a d20 and adding their spellcasting modifier. On success, the winner takes control of the spirit, while upon failure the spell ends.

This spell's maximum duration increases when you reach 5th level (8 hours), 11th level (1 day), and 17th level (1 week).

*Requires the deceased's name and knowledge of their face

\end{document}
