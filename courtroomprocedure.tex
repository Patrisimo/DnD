\documentclass{article}
\usepackage{amsmath, amssymb, fullpage, graphicx}
\usepackage{enumerate,lastpage}

\begin{document}
\begin{center}
Defense Attorney 101\\

Mia Fey
\end{center}

\section{Overview}
The Losokyo legal system is different from the legal systems you may have encountered in other locations. Here, we use the \emph{inquisitorial} system, where the prosecution and the defense work together to find the truth of the incident. At least, that's how it \textit{should} work, but with the hierarchy of the system (investigative rights lie with the Prosecutor's Office, and the Police Department works under the Prosecutor's Office) and social biases (we defense attorneys are often demonized for helping criminals), the fact is that nearly all defendants are found guilty. Of course, many of these defendants are actually guilty, but when an innocent person is imprisoned, not only have we failed that person, but we have also failed to bring the true culprit to justice.\\


Our job is further made difficult by the fact that the prosecution always holds the ultimate evidence of our client's guilt: the very fact that the crime occurred. It is not enough for us to muddy the waters and create doubt that our client did it: we must prove with evidence that our client is innocent without a doubt. This is typically done by finding the true culprit, and \textit{that} is typically done by getting the culprit to take the stand and pressuring them until they confess. In the meantime, however, our primary goal is to keep the trial going at all costs, for as soon as the prosecution completes an argument without an objection from the defense, our client will be found guilty.\\

In our legal system, trials are mediated by the judge and typically conducted by the prosecutor. The trial begins with the prosecution's opening statement, which leads into the prosecution's main argument. The prosecution will support their argument with physical evidence and/or with witness testimony. Notably, the prosecution is the only one who can call witnesses to the stand, however the defense may ask for someone's testimony. Witness testimony is our time to shine. After the witness makes their statements and the judge and prosecution ask their follow-up questions, we have the right to \textit{cross-examine} them. During this time there are two things we can do: we can \textit{press} a witness to elaborate on a statement they made, or we can \textit{present} evidence that contradicts their testimony. Be careful, however, as presenting irrelevant evidence or making obviously faulty arguments can wear on the judge's patience. Further, sometimes all you need is to press a statement in order to get new testimony. \\

As defense attorneys, we can never truly know whether our clients are guilty or innocent. All we can do is believe in them. When trying to logically figure out what happened, our first premise must always be that our client is innocent. Thus, when a witness gives testimony that incriminates our client, the witness must be lying or mistaken. However, it isn't enough for us to point out the contradiction, we must also explain what the contradiction \textit{means}. Why did the witness make that error? If you can answer that question, then you are one step closer to solving the case. Remember that there is a decent chance that the witness is the culprit, as (purely semantically) the culprit is always a witness. 

One final addendum, don't hesitate to ask your co-counsel for help! I'll always be there for you, and if you ask me what I think of the current state of the trial, I should be able to give some hints as to what seems fishy!\\


To recap:
\begin{itemize}
\item We must always believe in our clients until the very end
\item Once indicted, our client will be presumed guilty until proven innocent
\item Cross-examination is our main tool for proving our client's innocence
\item The culprit is a witness, try to get them on the stand
\item Any explanation is better than no explanation
\item Talk things over with your co-counsel - me!
\end{itemize} 

Best,\\

Mia Fey






\end{document}
