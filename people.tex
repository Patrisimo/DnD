
\section{People}
\subsection{Plegia}


\subsubsection{Broderick}
\label{people:tslil}
\begin{center}
\begin{tabular}{c c|c c}
STR & 13 & INT & 9\\
DEX & 18 & WIS & 16 \\
CON & 12 & CHA & 9 \end{tabular}\end{center}
\textit{Human Monk}\\
A fellblood bearing the Mark of Grima on the back of their right hand. Son of \linkto{people:validar} and Lide, he bears both the proper Heart and Blood of Grima to be the Fell Dragon's vessel and aid in its ressurection. Unwilling to sacrifice her son for this purpose, Lide took Broderick and fled to Ylisse on his tenth birthday. For their protection, she disguised herself as a relative, Aunt Milibette, and told Broderick that his parents were killed. When Broderick was 15, \linkto{religion:grima}, through the time travel process initiated by \linkto{people:bends} and \linkto{religion:naga}, attempted to take control of his body, however Broderick was too weak and instead lost all of his memories and gained the Mark, occasionally receiving visions of Grima's future. Lide, alarmed by the appearance of the Mark, gave him an earring, \linkto{items:lideearring}, to hide it. However, when Broderick reveals his visions to her, she realizes that he is not safe with her and leaves him in the care of the nearby monastery.

Broderick's childhood is as much a mystery now, to him, as it has ever
been. Beyond fleeting and shattered fragments of jagged rememberings,
none quite seeming real or ever lingering long enough to be processed
by the conscious mind, Broderick would not be able to describe the
circumstances of his youth any more than he would be able to explain
how he came to be in the care of his aunt Milibette. As far as he is
concerned, things have roughly always been this way.

For as far back as contiguous or even tangible memory extends,
Broderick had lived in a small hut with Hilda, nestled in the
foothills of a mountain where they tended to their meagre fields and
few livestock, all but isolated from the world. A day's journey in one
direction lead to a village too insignificant for even for the most
detailed of maps, a village of no particular import and of no
particular name. A few day's journey in another direction, up the
mountain lead to a monastery. It was for this monastery, the Monastery
of Highsleaux, and the mountain Ebpel Rulugenpud as the monks
insisted, that Milibette had brought Broderick here.

His story, as she would he had it, was that Broderick was once son to
a loving father and mother. Agen and Lide were their names, and they
cared for their son without thought for another. Through stout labour
and husbandry they ensured that all the needs of young Broderick were
met, and Agen was a skilled carpenter out of whose workshop many a toy
were produced. There, Milibette said, was were Broderick honed his
keen senses in the careful measuring and alignment of the structures
his father built, and the warding of the fields from small game. Life
was bliss, she said, but as is the capricious nature of the gods this
joy was torn maliciously from its stem before a flower it could
become, and the soil sown with the unspeakable crimes that it
witnessed.

Milibette spoke sparingly of this terminal frost, a marauding group of
Orcs attacked the village to which the homestead was adjoined,
methodically and deliberately paring all that had been brought to
blossom. She was taking care of Broderick while Agen and Lide were out
trading wares with the villagers. The orcs raided the home on their
way out, while the child and guardian quivered silently in the barn.

More of these events Broderick would never learn, and even such scant
recollections were hard to come by -- Milibette only hinting at them
on rare evenings during which some inner force of irreconcilable guilt
or sadness seemed to be consuming her. The same forces, she asserted,
held Brodericks memories captive -- unwilling to relinquish their icy
grasp lest Broderick's very person be rended asunder by the horrid
truth. Trauma, it would seem, was both an agent of terrible change and
one of sterilisation. The wound was unfathomably deep, but clean.

All of this was why the had to move. She collected what remained,
spared no comfort, and attempted to replant far away from all of this,
in the safety of anonymity and insignificance, under the careful watch
of the monks. And so Broderick came to be, again. The only hint that
any of this was real, aside from the yawning cavity in his memories,
was a simple earring. An earring, Milibette said, which had belonged
to Lide -- whoever she was.

Broderick would have let this all be, put it aside to the best of his
abilities, were it not for the visions. They came, infrequently at
first, manifesting as headaches before some transplanting of reality
occurred so believable that Broderick could scarcely reconnect with
the world upon their termination. He dared not speak of them for many
years, and when he finally did, his world was once more torn from its
roots.

Milibette announced that he was to be deposited in the care of the
monks, she could no longer tend to him and he would not flourish under
her care. He had barely organised his reality when he was forced into
a life of asceticism and study. Milibette moved to the village, but
Broderick was not permitted to leave the extended grounds of the
monastery all to visit her.

Many years passed in the monastery << more here >> but he never quite
seemed to grow in the way they wanted. Books had their uses, he
admitted, but he would often choose to spend his time alone, away, on
the outskirts of the grounds, testing his eyes against the landscape
as though one day they might be trained to become sharp enough to
finally notice the flaw in reality -- the seem at the very edge of the
world that would reveal to him what was behind it all. He lived to
perceive, and hoped that one day he would have the clarity to pierce
the miasma clouding his memories or the strangling darkness obscuring
his visions. One day he would be able to look upon his life and see,
see who he was and what it was all for.

\subsubsection{King Gangrel}
\label{people:gangrel}

\subsubsection{Validar}
\label{people:validar}
A human illusionist and head of the Grimleal. Years of religious loyalty to Grima and practice of dark magic has twisted his form, and so he now permanently dons an illusion to look like a tiefling to mask his horrid figure. The source of this illusion is his holy symbol, a \linkto{items:grimanecklace}, which he always wears under his cloak.

\subsection{Regna Ferox}

\subsubsection{Klob Rockbeard}
\label{people:daniel}
\begin{center}
\begin{tabular}{c c|c c}
STR & 18 & INT & 8\\
DEX & 14 & WIS & 10 \\
CON & 19 & CHA & 10 \end{tabular}\end{center}
\textit{Dwarf Barbarian}\\


\subsection{Ylisse}

\subsubsection{Aleneth}
\label{people:bends}
\begin{center}
\begin{tabular}{c c|c c}
STR & 10 & INT & 14\\
DEX & 12 & WIS & 9 \\
CON & 14 & CHA & 20 \end{tabular}\end{center}
\textit{Half-elf Warlock}\\

Identity taken of Exalt Lucina in order to disguise her presence in the current timeline. The daughter of Krom and ??? ten years from the start of the campaign, she became Exalt after the death of her mother, which immediately preceded the rebirth of \linkto{religion:grima}. With her world's destruction at hand, she and her companions manage to recover all but one of the gems (Argent) and perform the Awakening ritual. However, without the full complement of gemstones, \linkto{religion:naga} is unable to impart her full power to the \linkto{items:falchion}, instead sending Lucina and her companions back in time to shortly before the start of the campaign. 

Unbeknownst to Lucina, Grima interferes with this process, causing the travellers to all arrive in different times and locations, and attempts to take control of his vessel, \linkto{people:tslil}. However, Broderick's body is too weak to hold all of Grima's power, and instead he loses his memory and gains the Mark of Grima. Grima also created another timeline, initially without Naga's knowledge. 
\begin{itemize}
\item If Grima is sealed away in both timelines, then so too will he be sealed in the ``true'' timeline, as the three paths work to recombine. Lucina and the others will be allowed to return to their timeline.
\begin{itemize}
\item In this case, if Grima is \textit{killed} in either timeline, he will be killed in all of them
\end{itemize}
\item If Grima is awakened in both timelines and allowed to reach full strength, then Lucina and the party will be unable to return to their timline, and all will be lost.
\item If Grima is only sealed in one timeline, then Lucina and the party will be able to return to their timeline, however the events in the two timelines may have altered the state of affairs
\end{itemize}

Because the Awakening was performed without all the gemstones, Lucina also needed to offer her right hand as sacrifice, replacing some of the power of the missing gemstone, Argent. She has three main goals that she needs to accomplish in order to hopefully prevent the ressurection of Grima:
\begin{enumerate}
\item Prevent an assassination attempt on Krom, which later leads to his death
\item Prevent the assassination of Exalt Emmeryn
\item Prevent Plegia from obtaining the Shield of Seals and the gemstones.
\end{enumerate}

Lucina knows that her mother was killed by someone close to her, but does not know it was actually \linkto{people:tslil}.

\subsubsection{Dahlia Hawthorne}
\label{people:dahlia}

\subsubsection{Dick Gumshoe}
\label{people:gumshoe}
\begin{center}
\begin{tabular}{c c|c c}
STR & 15 & INT & 6\\
DEX & 9 & WIS & 6 \\
CON & 16 & CHA & 11 \end{tabular}\end{center}
\textit{Human}\\

\subsubsection{Emmeryn}
\label{people:emmeryn}


\subsubsection{Gregory Edgeworth}
\label{people:gregory}

\subsubsection{Judge}
\label{people:judge}


\subsubsection{Krom}
\label{people:chrom}



\subsubsection{Larry Butz}
\label{people:larry}
\begin{center}
\begin{tabular}{c c|c c}
STR & 11 & INT & 7\\
DEX & 14 & WIS & 6 \\
CON & 10 & CHA & 14 \end{tabular}\end{center}
\textit{Human}\\
Originally from the same gaucho community as \linkto{people:david}, Larry and San Juro were close childhood friends until Larry's family moved to Losokyo in 65 Mostyn. Notoriously excitable and lazy, he was infamous for getting caught up in unfortunate situations, to the point where a saying developed: ``When something smells, it's usually the Butz''. At the start of the campaign, Larry is dating Cindy Stone, a model.

\subsubsection{Manfred von Karma}
\label{people:mvonkarma}

\subsubsection{Maya Fey}
\label{people:mayafey}

\subsubsection{Miles Edgeworth}
\label{people:miles}

\subsubsection{Mia Fey}
\label{people:miafey}
\begin{center}
\begin{tabular}{c c|c c}
STR & 9 & INT & 17\\
DEX & 12 & WIS & 15 \\
CON & 11 & CHA & 16 \end{tabular}\end{center}
\textit{Human}\\



\subsubsection{Misty Fey}
\label{people:mistyfey}

\subsubsection{Mr. Briney}
\label{people:briney}

\subsubsection{Redd White}
\label{people:redd}
\begin{center}
\begin{tabular}{c c|c c}
STR & 12 & INT & 8 \\
DEX & 15 & WIS & 12 \\
CON & 12 & CHA & 20 \end{tabular}\end{center}
\textit{Human}\\
Currently the CEO of the information-gathering company Bluecorp, a company that works on projects ranging from detective work to journalism to foreign intelligence. However, all of Bluecorp's activities are with the aim of uncovering kompromat on individuals for the purposes of blackmail, a project that has been remarkably successful and makes Bluecorp, and by extension, Redd, all but untouchable by the law. \\
\\
While not a member of the Grimleal, Redd White was approached by the Grimleal during the Crusade to sell information to the Plegians in order to hamper the Ylisseans. While some contact is still maintained, Redd and his company currently work mostly independently from Plegia. \\
\\
One of Bluecorp's early successes outside of anti-Crusade work was the acquisition of the secret behind \linkto{events:dl6}, which it bought from \linkto{people:grossberg} and subsequently leaked to the media. This won him control over Grossberg as the police began to search for the source of the leak.\\
\\
Bluecorp employs a great number of people to perform its information gathering. These include his secretary \linkto{people:aprilmay}, a bandit captain Stella Artois, and close associate Himika Kazuro (\linkto{people:cyew}). \\
\\
A rather pompous man, Redd frequently makes up his own words or switches into Spanish. He collects magical artifacts, a hobby financed by his lucrative business. As such, he very well protected with warded items, and nearly unbeatable when on his own turf. 

\subsubsection{Robert Hammond}
\label{people:hammond}

\subsubsection{San Juro}
\label{people:david}
\begin{center}
\begin{tabular}{c c|c c}
STR & 9 & INT & 9 \\
DEX & 19 & WIS & 15 \\
CON & 12 & CHA & 14 \end{tabular}\end{center}
\textit{Human Gunslinger}\\
\\
San Juro was a gaucho in the southern planes, part of a small bison wrangling community. His wife was due with child. Due to the dangers of the wild, and the threats of bandit raids, he always carried his trusty pistol at his side. He lived a simple life until one day, a raid by the notorious Five Fists bandit crew upended everything. The crew shot his wife dead as they ransacked the town, and San swore revenge. Having shot a lackey dead, he picked up his pistol (begining his dual wielding trend), and tracked down the rest of the crew.

When he found them, however, they had already got got, and were lying bleeding or dead by the roadside. Before finishing off the wounded, San found that one of the crew had ran. Broken and defeated by his unfinished revenge, San became a hired gun, roaming the southern planes and the towns therein.


\subsubsection{Terry Fawles}
\label{people:fawles}

\subsubsection{Thalassa Gramarye}
\label{people:thalassa}

\subsubsection{Valerie Hawthorne}
\label{people:valerie}

\subsubsection{Winston Payne}
\label{people:wpayne}

\subsubsection{Yanni Yogi}
\label{people:yogi}
