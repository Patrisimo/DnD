\section{Bandits of Eagle Mountain}

Over dinner with Mia, the party learns that they have been assembled to try and restore faith in law and order in Losokyo. Mia gives a brief history of the rise and fall of the Losokyan legal system. If pressed, she can also describe \linkto{events:dl6} and \linkto{events:kidnapdh}.
Talking with people around town about the bandit problem, the party learns that the city has almost been under seige for the last year because of bandits. Trade caravans need to have substantial escorts, as the bandits always seem to know when conveys are scheduled to leave or arrive\footnote{Redd White's Bluecorp is behind this}. Losokyan militias have scoured the countryside looking for the bandit camp, but have not found any trace of one.

One particularly intrepid journalist, \linkto{people:lotta}, has been investigating rumors of a ``Pasquash'' in the forested mountains east of the city. In that process, she stumbled across wagon tracks leading to the edge of the ravine of Eagle River, in the Eagle Mountains. 

\subsection{Eagle Mountain}
High up in the mountains, the road leads to a cliff with a wooden bridge spanning the chasm to the other side. Smaller paths head further up the mountain on this side of the chasm. At the bottom of the chasm, 50 feet below, the Eagle River rages, although the area under the bridge on the far side is a rocky bank, about 10 feet above the river. 

To the south, a large cliff prevents vision beyond about 30 feet, although brilliant colored shapes can be seen occasionally peeking above the ridge. On the near-side cliff wall, about 20 feet down, there is a cave. 

\subsection{The Cave}
\textit{Music: Mahalo Trail or Aether Foundation Basement}\\
The cave quickly opens up into a large room, with a dozen statues lining the left and right walls and one placed near the middle of the room, apparently figures in the middle of various actions, mostly seeming to be in combat. Each statue has a couple glyphs written on its base. The statue in the center is a statue of King Primidux\footnote{DC 20 History or Religion to recognize}, facing to the right with his hands, and sword, facing the left. On the opposite wall, a message, encoded in glyphs, is engraved in marble that has been carefully inserted into the stone wall. To either side of the message, a pair of magical flames give the room a dim light, and below each flame is a shield-bearing statue\footnote{think Big Shield Gardna from Yu-gi-oh}, for 27 statues in total\footnote{because this scenario occurred on a player's 27th birthday}. Along the sides of the room, a sequence of 30 glyphs occur on either side, both in the same order\footnote{It's the alphabet and ,.!?}. In the center of the room, there are 25 depressions in the ground that match the shape of the bases of the statues. In the center of that circle, there is a smooth orb. The 25 statues are all of the members of \linkto{events:bcc}; characters from respective locations might recognize one or a couple of the statues from national legend.

\subsection{The Puzzle}
To proceed, the statues must be placed in their correct locations in the circle. When in the correct location, the glyphs on the statues when read counterclockwise starting with King Primidux replicate the message engraved on the front wall. A high intelligence roll or smart player might realize the glyphs on the walls are meant to be an alphabet, which allows them to decode the message:
\begin{center}
\includegraphics[scale=0.5]{combats/alphabet.PNG}\\
- The ``Alphabet''\\

``abcdefghijklmnopqrstuvwxyz,.!?''\\
- The alphabet, decoded\\

\includegraphics[scale=0.5]{combats/message.PNG}\\
- The message, divided into the statues \\

``Those who inherit our will shall shine the middle''\\
- The message, decoded
\end{center}

The message is actually backwards, and so reading \textit{clockwise} reveals the meaning of the message. When the statues are placed in the right order, a door underneath the engraving opens, however this door is alarmed. The alarm will be disabled if a bright light is shone upon the orb in the center; when that happens, the orb will emit its own bright light, and small flames will erupt on each of the statues in an appropriate location (arrowtip, center of shield, etc).

\subsection{The Bandit Hideout}
\textit{Music: Po Town}\\
\includegraphics[scale=0.5]{combats/banditcave.PNG}\\
The bandits' initial positions are as follows:
\begin{itemize}
\item Five Bandits are asleep in the quarters
\item One Bandit is patrolling the main hallway (outside of the quarters)
\item Three Thugs are meeting in the meeting room
\item Four Bandits are eating in the Great Hall
\item Stella Artois (\textit{Bandit Captain}) is in her study
\end{itemize}
If the alarm is tripped, all the bandits will move to the Great Hall to ambush the party when they walk in, except one, which will be in the watchroom. As soon as a fight starts, the noise will attract the attention of the other bandits who, depending on the situation, will either come help or will coalesce in the Great Hall.\\
\\
Ambush formation involves half of the bandits climbing the stairs in the Great Hall and standing above the entrance they expect the party to enter from, while the other half hide behind the tables (DC 20 Perception). Stella Artois stands outside her study to parlay with the intruders, however if talks go south she will signal for the crossbows to fire upon the party, possibly starting a surprise round. \\
\\
Each bandit drops 3d6 gold in addition to their equipment. Stella will fight to the death, however when Stella drops to low health, or there are at most 2 bandits remaining, she will cry ``No... I can't be captured!'', although once at low health she might accept an offer of surrender. If interrogated, she will reluctantly reveal that she became a bandit captain at the command of someone else, in order to protect her son Fekete, who was kidnapped. If the party loots the treasure room, they will find:
\begin{itemize}
\item 2700 cp
\item 900 sp
\item 70 gp
\item Large pound cake (5gp)\footnote{Again for birthday purposes}
\item Rabbit Fur Ribbon (25 gp)
\item Silk Robe (25 gp)
\item Leather Vest (25 gp)
\item 3x Onyx (50 gp)
\item Box of China (100 gp)
\item Cloak of Protection
\item \linkto{items:peeko}
\item Ring of Water Walking
\end{itemize}

If the party investigates Stella's office, they will find a ledger of recent bandit attacks going back about three months, as well as a couple bowls from the Borscht Bowl Club.\\
\\
If the party brings back proof of clearing the bandit camp and presents it to a guard, they will be brought before the Mayor and receive 2d4 * 100 gold as reward. The Mayor will ask the party if they managed to retrieve any of the stolen goods, and if they did, will offer to buy them at value and return them to their rightful owners. Depending on how this interaction goes, the Mayor may be left with a favorable view of the party, and may even (secretly?) inform the Exalt of their success. Alternatively, if the party did not investigate any of the clues, the Mayor will mention this and criticize them for not investigating further.
\begin{center}
``I guess that is why we pay some people to fight and others to think. Very well, I will contract some investigators to look into this business. Truly, thank you for the deed you have done us, but we shall take care of it from here.''
\end{center}
Some of the problems the party could realize:
\begin{itemize}
\item Where did the cavern come from?
\item What did the symbols in the antechamber mean? (Force the stone to glow)
\item How did the bandits enter without setting the puzzle?
\item How did the bandits bring their goods into the cave?
\item How did the bandits access the cave without leaving a rope?
\item Why were the wolves on the trail so vicious?
\end{itemize}
If pressed, the mayor will allow the group one month to prove their investigative prowess. Regardless, this ends the scenario, and begins the exposition of the next section.